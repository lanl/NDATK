% Template larmemos -- 13 Jan 99

\documentclass[12pt]{lamemo}
\usepackage{times}
\usepackage{amssymb}
\usepackage{listings}
\usepackage{graphicx}
\DeclareGraphicsExtensions{.jpg,.pdf,.mps,.png}
\DeclareGraphicsRule{*}{mps}{*}{}

\usepackage{color}
\definecolor{amber}{rgb}{1,.5,0}
\usepackage[normalem]{ulem}

%\usepackage{newcent}
%\pagestyle{secret}             % use this pagestyle for SRD
%\pagestyle{unclassified}       % use this pagestyle for U
%\pagestyle{official}           % use this pagestyle for OUO
\pagestyle{unmarked}           % use this pagestyle for no marking
% Fixed information
% All lines are required.
\divisionname{Computational Physics}	% center, project, or divison name 
\groupname{XCP--5}		% organization number and/or name
\phone{7-5341, 5-2879}		% sender phone, FAX number
\fromms{Mark G. Gray, F663}	% sender initials, mail stop
\originator{mgg}		% memo originator
\typist{mgg}			% memo typist

% Information on this memo
% The \toms, \refno, and \subject commands are required.

\toms{Robert C. Little, B218} % recipient initials, mail stop

\refno{XCP--5:15--005(U)} % reference number 

\subject{The Nuclear Data Access Tool Kit}

\date{November 14, 2014}

% Optional information:
\adc{dkp}
% \thru{nobody, MSnowhere}       % person(s) to send memo through
% \cy{aaa\\bbb}                  % copy list
\distribution{Stephanie C. Frankle, B259\\
Albert C. Kahler, B214\\
Kevin G. Honnell, F663\\
Jon A. Dahl, D409\\
Terry R. Adams, F605\\
Steven D. Nolen, F605\\
Jeremy E. Sweezy, A143\\
XCP--5 File}    % distribution list
% \enc{aaa\\bbb}                 % list of enclosures
% \encas                         % Enc. a/s 
% \attachments{aaa\\bbb}         % list of attachments
% \attachmentas                  % attachment as stated
% \attachmentsas                 % attachments as stated
% Text of the memo

\newcommand{\ndatk}{\texttt{ndatk}}
\newcommand{\MCATK}{\texttt{MCATK}}
\newcommand{\NDI}{\texttt{NDI}}
\newcommand{\zaid}{\texttt{SZAID}}

\begin{document}
\lstset{language=C++}

\maketitle			% make memo header

\section{Summary}
% I want to tell Bob that...
The Nuclear Data Access Tool Kit, which provides continuous energy
neutron libraries corresponding to multigroup neutron libraries, is
now available for production use.  The code and \texttt{lanl2006},
\texttt{endf71x}, and \texttt{mtmg08} continuous energy neutron data
libraries are available on the open and secure X-LAN, open and secure
HPC network, and on Livermore's Sequoia computer.

\section{Background}

On December 10, 2012 the Nuclear Data Team and its management met with
the Monte Carlo Application Tool Kit\cite{adams14} (\MCATK) team and
its users to discuss the translation of multigroup neutron table
identifiers into continuous energy neutron table identifiers for the
comparison of transport methods.  At two subsequent meetings (December
20, 2012 and January 23, 2013) we decided that grouping continuous
energy tables in multigroup like libraries was sufficient, and that
this could best be done through an access interface code, the Nuclear
Data Access Tool Kit (\ndatk).

On May 30, 2013 I demonstrated the alpha version of \ndatk, with
support for the single temperature continuous energy neutron library
\texttt{lanl2006}, and provided \ndatk\ alpha to \MCATK\ developers
the next day.  On November 15, 2013 I demonstrated the beta version of
\ndatk, with the addition of the multi-temperature continuous energy
neutron library \texttt{endf71x}, and provided \ndatk\ beta to
\MCATK\ developers on January 17, 2014.  On May 18, 2014 I
demonstrated the production version of \ndatk\ with the addition of
the multi-temperature neutron partial library \texttt{mtmg08}, standard
library install directories, and automatic path search for data.
After passing \MCATK\ acceptance tests, the code was distributed to
LANL platforms September 3, 2014 and LLNL platforms November 13, 2014.

This memo describes the completion of the initial production version of \ndatk.
\newpage

\section{Discussion}

The planning meetings produced the following list of desirable
features for \ndatk:
\begin{enumerate}
\item Translate from multigroup to continuous energy \zaid s,
\item Access through polyvalent program pattern\cite[p. 281]{raymond04},
\item Provide Nuclear Data Interface\cite{campbell98} (\NDI)
  multi-temperature features,
\item Provide \NDI\ \zaid\ translation features,
\item Support curated data,
\item Support continuous energy libraries.
\end{enumerate}

Because of the limited resources available to the \ndatk\ project, I
decided to:
\begin{itemize}
\item analyze the desirables to produce a (hopefully smaller) set of
  requirements, 
\item design from existing code and data formats whenever possible,
\item prototype extensively using an interpreted language
\item prioritize functionality and stage delivery,
\end{itemize}
as a means to both manage the schedule and mitigate the risks.

\subsection{\hspace{-1.5em}Analysis of Desirables: }
Although the initial conception of \ndatk\ focused on translating from
multigroup to continuous energy \zaid s (Desirable 1), I focused on
the creation of named libraries (Desirable 6) as a means to match the
\NDI\ specification of library, \texttt{SZA}, and possibly temperature
(Desirable 3) for table access, obviating the need for Desirable 1 and
further providing the means for both \zaid\ translation (Desirable 4)
and data curation (Desirable 5).  And although the initial desire was
for a multi-language API (Desirable 2), I focused on delivering a
\texttt{C++} API since that's the language used by the primary
customer (\MCATK).

Based on this analysis the requirements for the first production
version of \ndatk\ are:
\begin{enumerate}
\item Support continuous energy libraries,
\item Access through \texttt{C++},
\item Provide \NDI\ multi-temperature features,
\item Provide \NDI\ \zaid\ translation features,
\item Support Curated Data.
\end{enumerate}


\subsection{\hspace{-1.5em}Design from Existing Code and Data: }
Figure~\ref{fig:frame} shows the commanded information frame, similar
to the one used for \NDI, chosen to address these requirements.
\begin{figure}\centering
\hrulefill \\
\includegraphics[height=3in,width=5in]{ndatk.10}
\caption{The commanded information frame\cite[p. 215]{jackson01} of
  \ndatk.  When the user's code makes a request to \ndatk\ (UC!R), it
  responds by consulting the nuclear data sources available to it
  (ndatk!ND) and returning user data (ndatk!UD) that fulfills the
  request.  The requirement (R) for \ndatk\ is to make the user's data
  (UD) correspond to the nuclear data (ND) and report its
  status.} \label{fig:frame} \hrulefill\\
\end{figure}
The design of \ndatk\ is thus levelized, with the Application
Programming Interface (API) providing data surrogates visible to
users, the Finite State Machine (FSM) layer maintaining state, various
Readers accessing nuclear data files, and a Utilities layer providing
various general algorithms.  The fundamental requirement (R) for
\ndatk\ is to respond to a user's request, say for data from a named
library by \texttt{SZA} and temperature, by consulting its available
nuclear data files and returning to the user the requested data's
location and a status indicator of success.

The data handles\cite{gray04} of \NDI, which act as opaque pointers to
its GENeral DIRectory (\texttt{gendir}) and multigroup neutron data
files translate directly into the design of \texttt{C++} interface
objects as resource handles in \ndatk.  The cross(X) Section
DIRrectory\cite[Appendix F]{briesmeister00} (\texttt{xsdir}) of
\texttt{MCNP}\cite{briesmeister00} provides the basic format and
initial implementation of the Extended cross(X) Section DIRectory
(\texttt{Exsdir}) file in \ndatk.  The record-jar
format\cite[p. 116]{raymond04} used by \NDI\ multigroup neutron
files\cite{campbell98a} provides the template for \ndatk's data file
formats.

The \ndatk\ package (\texttt{C++} namespace)
contains five class and three function modules as shown in
Figure~\ref{fig:package}.  It relies only on a standard
\texttt{C++03} (ISO/IEC 14882:2003) compiler and the Standard Template
Library.
\begin{figure}[h!]\centering
\hrulefill\\
\includegraphics[height=3in,width=5in]{package.pdf}
\caption{UML Package Diagram for \ndatk. The package consists of five
  class and three function modules, and relies solely on
  the \texttt{C++} Standard Template Library.} \label{fig:package}
\hrulefill\\
\end{figure}

The three interface classes, shown in Figure~\ref{fig:class} provide
the API representation of Extended cross(X) Section DIRectories
(\texttt{Exsdir}) data which map continuous energy \zaid s to table
meta-data including temperature, file name, and line offset;
continuous energy neutron libraries (\texttt{Library}) data
which multi-map \texttt{SZA} and temperature to \zaid; and chart of
the nuclides (\texttt{Chart}) data which map elements and isotopes to their
(scalar) physical properties.
\begin{figure}\centering
\hrulefill\\
\includegraphics[height=6in,width=5in]{class.pdf}
\caption{UML Class Diagram for \ndatk.  The \texttt{Exsdir},
  \texttt{Library}, and \texttt{Chart} interface classes all derive
  their provenance features from the \texttt{CuratedData} abstract
  base class.  A \texttt{Chart} object uses an \texttt{Exsdir} object
  to find its data at its creation, while the \texttt{Library} object
  maintains a reference to the \texttt{Exsdir} object to find table
  data throughout its lifetime.  An \texttt{Exsdir} object owns a
  \texttt{Finder} object which it uses to search for files in standard
  places.  The public methods shown are those necessary to satisfy
  \ndatk's three use cases.  For more details see
  \emph{\ndatk\ Technical Documentation}.} \label{fig:class}
\hrulefill\\
\end{figure}
The \texttt{CuratedData} abstract base class provides each of the data
interface classes with provenance features.  The \texttt{Finder} class
provides default file location and search behavior for the
\texttt{Exsdir} class.

In order to guarantee provenance of its data, the \ndatk\ production
release includes the data libraries shown in Table~\ref{tbl:data}.
\begin{table}\centering
\hrulefill\\
\caption{Data libraries in \ndatk.  The production release of
  \ndatk\ contains three data libraries: a set of single temperature
  continuous energy tables which correspond to the \texttt{lanl2006}
  single temperature multigroup library, a set of multi-temperature
  continuous energy tables which come from the same evaluation as the
  \texttt{mendf71x} single temperature multigroup library, and a set
  of multi-temperature continuous energy tables which come from a
  subset of the evaluations used for the \texttt{mtmg08}
  multi-temperature multigroup library.} \label{tbl:data}
\begin{tabular}{rrr|l}\hline
CE Library & Isotopes & Temperatures & MG Library \\ \hline
lanl2006 & 180 & 1 & $\approx$ lanl2006 \\
endf71x & 423 & 7 & $\supset$ mendf71x \\
mtmg08 & 63 & 10 & $\subset$ mtmg08 \\ 
\end{tabular}\\
\hrulefill
\end{table}

\subsection{\hspace{-1.5em}Prototype extensively: }
Table~\ref{tbl:code} shows the line counts for the prototype and
production code.  The design was prototyped extensively using
\texttt{Python} for rapid analysis of many different configurations, as
evidenced by the large number of lines of prototype code.
\begin{table}\centering
\hrulefill\\
\caption{Code line count for \ndatk.  The production release of
  \ndatk\ contains a code library, 275,343 unit tests and 3 use cases
  which were designed and prototyped based on \texttt{Python} code.
  The basic \texttt{COCOMO}\cite{boehm81} model predicts a 4
  person-month effort for the total amount of production code, which
  compares favorably to the actual 1/4 person times 17 month
  effort.} \label{tbl:code}
\begin{tabular}{lr|l}\hline
\ndatk & Lines of Code & Language \\ \hline
Prototype Code & 6442 & \texttt{Python} \\ \hline
Library Code & 830 & \texttt{C++} \\
Unit Test Code & 316 & \texttt{C++} \\
Use Case Code & 131 & \texttt{C++} \\ \hline
Total Production Code & 1277 & \texttt{C++}
\end{tabular}\\
\hrulefill
\end{table}
As a result this upfront experimentation, the production code tightly
focuses on only the necessary functionality.

\subsection{\hspace{-1.5em}Prioritize Functionality and Stage Delivery:}
The \ndatk\ project focused on satisfying three use cases:
\begin{enumerate}\setcounter{enumi}{-1}
\item Single Temperature Library Access: create \texttt{Exsdir} by
  name; create \texttt{Library} by name and \texttt{Exsdir}; retrieve
  isotope file name and line number offset by isomer name.  See
  Appendix~I, page~\pageref{app:use0}. 
\item Calculate Elemental Atomic Weight Ratios (AWR) from Isotopic
  Data: create \texttt{Exsdir} by name; create \texttt{Chart} by name
  and \texttt{Exsdir}; calculate elemental atomic weight ratio from
  isotopic abundances and atomic weight ratios.  See Appendix~II,
  page~\pageref{app:use1}.
\item Multi-temperature Library Access: create \texttt{Exsdir} by
  name; create \texttt{Library} by name and \texttt{Exsdir}; retrieve
  isotope file name and line number offset by isomer name; for each
  isomer temperature in library retrieve table name.  See
  Appendix~III, page~\pageref{app:use2}.
\end{enumerate}
The first use case was satisfied by the alpha release, while the
second and third were first partially satisfied by the beta release
and then fully satisfied by the final release.

\begin{table}\centering
\hrulefill\\
\caption{Release history of \ndatk\ requirements and use cases.} \label{tbl:hist}
\begin{tabular}{l|ccc}\hline
Requirements & alpha & beta & final \\ \hline 
1. CE libraries & \texttt{lanl2006} & \texttt{lanl2006} &
\texttt{lanl2006} \\
       &  &
\texttt{endf71x} &
\texttt{endf71x} \\
             &  & & \texttt{mtmg08}\\ 
2. \texttt{C++} API & \textcolor{green}{\checkmark} &
\textcolor{green}{\checkmark} & \textcolor{green}{\checkmark}
\\ 3. Multi-temperature & \textcolor{red}{X} &
\textcolor{amber}{\checkmark} & \textcolor{green}{\checkmark}
\\ 4. \zaid\ translation & \textcolor{red}{X} &
\textcolor{green}{\checkmark} & \textcolor{green}{\checkmark}
\\ 5. Curated data & \textcolor{red}{X} & \textcolor{amber}{\checkmark} &
\textcolor{green}{\checkmark} \\ \hline Use Cases & & & \\ \hline
0. Single Temp. Library & \textcolor{green}{\checkmark} &
\textcolor{green}{\checkmark} & \textcolor{green}{\checkmark}
\\ 1. Calculate AWR & \textcolor{green}{\checkmark} &
\textcolor{green}{\checkmark} & \textcolor{green}{\checkmark}
\\ 2. Multi-temp. Library & \textcolor{red}{X} &
\textcolor{green}{\checkmark} & \textcolor{green}{\checkmark} \\
\end{tabular}\\
\hrulefill
\end{table}

The production installation of \ndatk\ includes both the code library
and header files, and the data library and associated tables.
Table~\ref{tbl:hist} summarizes the release history of \ndatk\ and the
requirements and use cases they satisfied.

\clearpage
\section{Conclusion}
The Nuclear Data Access Tool Kit is now available for production use.
The production version supports continuous energy neutron libraries
through a \texttt{C++} API, with \NDI\ like multi-temperature and
\zaid\ translation features.  All of the data available through
\ndatk\ is curated.  It satisfies all of its initial production
requirements and use cases.

The \ndatk\ project was finished on budget despite an extended
duration due to both initial and mid-cycle funding uncertainties.
Much of its success is attributable to the use of \texttt{Python} as a
rapid prototyping language which permitted rapid testing and refinement of
ideas and maintained project continuity throughout funding uncertainties.

The data libraries delivered with \ndatk\ meet the immediate
continuous energy data needs\cite{lee13}; we anticipate releasing
\texttt{mt71x}, based on the ENDF/B-VII.1 evaluations, as both
multi-group (\NDI) and continuous energy (\ndatk) multi-temperature
libraries.

Several enhancements could be added to a future update of \ndatk,
including: 
\begin{itemize}
\item Flat \texttt{C/Fortran} interface (Desirable 2)
\item Command line interface (Desirable 2)
\item Temperature match policy in Library
\item \texttt{SZA} and \zaid\ search policies in Library
\item Physical Constants class
\end{itemize}
The addition of a \texttt{C/Fortran} API would satisfy the polyvalent
program desirable and make it easier for codes other than \MCATK\ to
use \ndatk; a command line interface similar to \NDI's
\texttt{ndicl} would provide some scripting capabilities to the data
interface that has proved surprisingly useful in the case of
\texttt{ndicl}.  The current version of \ndatk\ returns the nearest
temperature to the one specified (similar to \NDI); a user define-able
policy could be added to the \texttt{Library} class to specify how
temperatures are matched.  Similarly, in the current version the
\texttt{SZA} or \zaid\ must match exactly some entry in the library; a
user definable policy could be added to the \texttt{Library} class to
permit substitution of other data if an exact match is not found.
Finally, although \ndatk\ does provide a handful of physical constants
in cross section data specific units, a \texttt{PhysicalConstants}
class could be added to \ndatk\ that would provide curated physical
constants in user specified units.

\newpage
\begin{thebibliography}{99}

\bibitem{adams14} T. Adams et al., \emph{Monte Carlo Application
  ToolKit (MCATK)}, Ann. Nucl. Energy (2014),
  http://dx.doi.org/10.1016/j.anucene.2014.08.047

\bibitem{raymond04} Eric S. Raymond, \emph{The Art of \texttt{UNIX}
  Programming}, Addison-Wesley, 2004

\bibitem{campbell98} J. M. Campbell et al., \emph{A Nuclear Data
  Interface for the ASCI Codes}, LAUR--98--5429, October 1998

\bibitem{jackson01} Michael Jackson, \emph{Problem Frames},
  Addison-Wesley, 2001

\bibitem{gray04} Mark G. Gray, \emph{\NDI\ 2.0 Interface
  Specification: Gendir and Multi-group Handles}, CCN--12:04--05,
    February 19, 2004

\bibitem{briesmeister00} Judith F. Briesmeister, Ed.,
  \emph{\texttt{MCNP} ---A General Monte Carlo N-Particle Transport
    Code}, LA--12709--M, March 2000 

\bibitem{campbell98a} J. M. Campbell and R. C. Little, \emph{A New
  Multigroup Library Format for ASCII Text Files}, XCI:98--77, May 22,
  1998
 
\bibitem{boehm81} B. Boehm, \emph{Software Engineering Economics},
  Prentice-Hall, 1981 

\bibitem{lee13} M. Beth Lee, \emph{Multi-temperature Continuous Energy
  Data for \texttt{MCATK}}, XCP--5:13--035(U), October 16, 2013

\end{thebibliography}

\footnotesize
\newpage
\section{Appendix~I: \texttt{use0\_Library.cc}}\label{app:use0}
\lstinputlisting{../../test/use0_Library.cc}
\newpage
\section{Appendix~II: \texttt{use1\_Chart.cc}}\label{app:use1}
\lstinputlisting{../../test/use1_Chart.cc}
\newpage
\section{Appendix~III: \texttt{use2\_Library.cc}}\label{app:use2}
\lstinputlisting{../../test/use2_Library.cc}

\end{document}

