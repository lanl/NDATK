% Template larmemos -- 13 Jan 99

\documentclass[12pt]{lamemo}
\usepackage{times}
\usepackage{amssymb}
\usepackage{listings}
\usepackage{graphicx}
\usepackage[all,cmtip]{xy}

\DeclareGraphicsExtensions{.jpg,.pdf,.mps,.png}
\DeclareGraphicsRule{*}{mps}{*}{}

\usepackage{color}
\definecolor{amber}{rgb}{1,.5,0}
\usepackage[normalem]{ulem}

%\usepackage{newcent}
%\pagestyle{secret}             % use this pagestyle for SRD
%\pagestyle{unclassified}       % use this pagestyle for U
%\pagestyle{official}           % use this pagestyle for OUO
\pagestyle{unmarked}           % use this pagestyle for no marking
% Fixed information
% All lines are required.
\divisionname{Computational Physics}	% center, project, or divison name 
\groupname{XCP--8}		% organization number and/or name
\phone{7-5341, 5-2879}		% sender phone, FAX number
\fromms{Mark G. Gray, F663}	% sender initials, mail stop
\originator{mgg}		% memo originator
\typist{mgg}			% memo typist

% Information on this memo
% The \toms, \refno, and \subject commands are required.

\toms{Joann M. Campbell, T087\\
      C. J. Solomon, F663\vspace{1mm}} % recipient initials, mail stop

\refno{XCP--8:18--006(U)} % reference number 

\subject{\texttt{ndatk} Element Support}

\date{December 13, 2017}

% Optional information:
\adc{dkp}
% \thru{nobody, MSnowhere}       % person(s) to send memo through
% \cy{aaa\\bbb}                  % copy list
\distribution{
Terry R. Adams, F605\\
Steven D. Nolen, F605\\
Jeremy E. Sweezy, A143\\
Travis J. Trahan, F663\\
Christopher J. Werner, F663\\
Austin P. McCartney, A143\\
Morgan C. White, F663\\
Brendan K. Krueger, T087\\
Aaron C. Koskelo, B259 \\
XCP--8 File}    % distribution list
% \enc{aaa\\bbb}                 % list of enclosures
% \encas                         % Enc. a/s 
% \attachments{aaa\\bbb}         % list of attachments
% \attachmentas                  % attachment as stated
% \attachmentsas                 % attachments as stated
% Text of the memo

\newcommand{\ndatk}{\texttt{ndatk}}
\newcommand{\MCATK}{\texttt{MCATK}}
\newcommand{\NDI}{\texttt{NDI}}
\newcommand{\zaid}{\texttt{SZAID}}
\newcommand{\MCNP}{\texttt{MCNP}}
\newcommand{\mcm}{\textcolor{green}{\checkmark}}
\newcommand{\mxm}{\textcolor{red}{$\times$}}
\begin{document}
\lstset{language=C++}

\maketitle			% make memo header

\section{Summary}
% I want to tell CJ that...

The Nuclear Data Access Tool Kit has been updated to resolve element
names into the elemental or constituent isotopic tables available in
the selected continuous energy library and consistent with the
calculated elemental tables in corresponding multigroup library.  This
extension was built from tool kit components which provide complete
information about what is present in the library versus what is
present naturally and an example for possible future capability,
including named material components.

\section{Background}

As \texttt{ENDF/B} evaluations moved from \texttt{VI} to \texttt{VII},
they moved from less precise elemental data to more precise nuclide
data.  Although at the time the nuclear data team recommended that
users expand elements in input files into their constituent natural
isotopes in proportion to their abundances, we found multigroup users
instead replacing elements with their most abundant isotope.

In January 2009 the nuclear data team developed \texttt{ndi\_zaid.py}
to calculate multigroup elemental tables\cite{gray09} that addressed
this problem; the multigroup \texttt{mendf71x} and \texttt{mt71x}
libraries deployed twenty-five calculated elemental tables.

In November 2014, I released the Nuclear Data Access Tool Kit
(\ndatk)\cite{gray14}, which groups continuous energy data into
logical libraries that match multigroup \NDI\ data libraries as
closely as possible.  Currently \ndatk\ (1.0.2) simply omits the
elementals calculated for multigroup.  Consequently, continuous energy
users have reverted to the bad habit of replacing elements with their
most abundant isotope.
 
In November 2015, I examined possible solutions to the elemental match
problem\cite{gray15}.  In October 2017, Joann Campbell made available
funds to implement a solution in \ndatk. This memo documents the
implementation of one of those solutions in \ndatk\ 1.1.0.

\newpage

\section{Discussion}

The \ndatk\ code library\cite{gray14} provides an interface to
continuous energy nuclear data tables which groups them into logical
libraries designed to match the multigroup energy nuclear data
libraries provided by \NDI.  Recent \texttt{ENDF/B} evaluations have
been parallel processed by the Nuclear Data Team to make this match
easy, with the notable exception of calculated multigroup elementals.

An elemental is a table containing kinematic (e.g. microscopic
reaction cross section) and kinetic (e.g. microscopic reaction energy)
data for the interactions of neutrons incident on an element target as
if the element were composed of a single kind of nuclide.  Although
the elemental kinematic values provide a good average representation
of the actual scattering physics, the elemental kinetic values require
the (a priori unknown) incident neutron flux to provide a good average
representation of the actual energetic physics.  For multigroup tables
the weight function provides an obvious estimate of the incident
neutron flux\cite[Eq.~3]{gray09}.

In a previous memo\cite{gray15} I examined options to match multigroup
elementals with continuous energy data.  Option~0, which required
continuous energy users to manually expand elements in their input,
and option~1, which required both multigroup and continuous energy
users to manually expand elements in their inputs, have failed.
Option~2, which requires the calculation of continuous energy
elementals, remains infeasible both for the lack of a suitable neutron
flux estimate and the incommensurate nuclide and reaction dependent
energy grids.  Option~4, which would automatically expand both
elements and named materials for both multigroup and continuous energy
libraries, is beyond the scope of the current work.  This memo
documents the implementation of option~3.

\begin{figure}\centering
\[
\xymatrix{
  & X \ar[ddl]_{\textcolor{red}{u}} \ar[ddr]^{\textcolor{red}{u}} & \\
  & L \ar[dl]^{\textcolor{blue}{h}} \ar@{<=>}[dr] \ar[u]_{\subset} & \\
  C \ar[rr]^{\subset} & & M \\
  & E \ar@{<=>}[ul] \ar[ur]_{\textcolor{blue}{d}} &
}
\]
\caption{Relations among Evaluations ($E$), Continuous Energy Libraries
  ($C$), Multigroup Libraries ($M$), Legacy Input ($L$), and Extended Input
  ($X$) in Option~3.  Continuous energy libraries are a subset of
  multigroup libraries, only continuous energy data faithfully matches
  evaluated data, the \textcolor{blue}{data team} calculates elemental
  tables for multigroup data to support legacy input while
  \textcolor{blue}{host codes} map legacy input into continuous energy
  data, and \textcolor{red}{users} map extended input into continuous
  energy and multigroup libraries.} \label{fig:3}
\end{figure}

The diagram in Figure~\ref{fig:3} summarizes the relationships among
nuclear data evaluations ($E$), continuous and multigroup energy
processed tables ($C$, $M$), and user input ($L$, $X$) in option~3.
The key to this option is to match the functionality of the existing
stand-alone code that calculates elementals as a weighted average of
isotope tables for inclusion in a multigroup library release:
\[
\textcolor{blue}{d}:E \rightarrow M \doteq \mathtt{ndi\_zaid.py},
\]
with new inline \ndatk\ calls that map element names to a mixture of
isotope tables in a related continuous energy library:
\[
\textcolor{blue}{h}:L \rightarrow C = \left\{ \begin{array}{l}
    \mathtt{ndi\_atom\_comp\_of} \\
    \mathtt{ndi\_mass\_comp\_of}
    \end{array} \right. .
\]
When the continuous energy transport host uses the natural isotopic
mix for an element, it calculates, with higher fidelity, the average
quantities found in the multigroup elemental tables.  

The \texttt{Chart} class in the current production release of \ndatk\
(1.0.2) already provides the natural abundance of isotopes by element
as illustrated in its release memo\cite[Appendix~II]{gray14}.  This
capability is encapsulated in this new release in two convenience
member functions:
\begin{description}
\item[\texttt{Chart::atom\_comp\_of}: ] natural atomic abundance of
  isotopes in element
\item[\texttt{Chart::mass\_comp\_of}: ] natural mass abundance of
  isotopes in element
\end{description}
that return a \texttt{map<int, double>} of the canonical SZA of
isotopes to their atomic or mass abundance in the element according to
the curated \texttt{Chart} data.

Providing analogous methods using the \texttt{Library} class raises
design tensions among maintaining a levelized design\cite{lakos96},
using tool kit components, maintaining existing\cite{gray14}
abstractions and encapsulations, and satisfying user requested new
features.  Simply adding analogous methods to the \texttt{Library}
class makes it dependent on the \texttt{Chart} class through only two
of its many member functions.  Adding methods to a new class dependent
on both \texttt{Chart} and \texttt{Library} keeps them independent,
but at the expense of a feature poor (at least in option~3) new
Level~4 class.  Adding analogous Level~4 functions to those in
\texttt{Chart} leads to redundant, complex\cite{wiki} ($\mathbf{M} =
7$), code.  Fortunately, these competing concerns can be resolved by
using the callback levelization technique.

Lakos defines the \textbf{Callback Levelization Technique} as:
\begin{quote}
  Using client supplied functions that enable lower-level subsystems
  to perform specific tasks in a more global context.\cite{lakos96}
\end{quote}
where, in the case of this \ndatk\ extension:
\begin{description}
\item[client supplied functions: ] \texttt{Chart::atom\_comp\_of}, 
  \texttt{Chart::mass\_comp\_of},
\item[lower-level subsystem: ] \texttt{Library::comp\_of},
\item[global context: ] \texttt{ndi\_atom\_comp\_of}, 
  \texttt{ndi\_mass\_comp\_of}. 
\end{description}
Here the callback keeps \texttt{Chart} and \texttt{Library} independent
at level~3, respects encapsulation of \texttt{Chart} informative
functions, eliminates potential atom and mass calculation redundancy in
\texttt{Library}, reduces complexity, and supports future expansion
with a \texttt{Chart}-like named material class.

The element expansion functionality is encapsulated in two level~4
convenience functions:
\begin{description}
\item[\texttt{ndi\_atom\_comp\_of}: ] library atomic abundance of
  isotopes in element,
\item[\texttt{ndi\_mass\_comp\_of}: ] library atomic abundance of
  isotopes in element,
\end{description}
which use the \texttt{Chart} composition functions as callbacks in the
\texttt{Library} composition function.

The \texttt{Chart::atom\_comp\_of} and \texttt{Chart::mass\_comp\_of}
methods were exhaustively unit tested against the data in the mock
four element, eight nuclide \texttt{early\_universe} ``Chart of the
Nuclides''.  The \texttt{Library::comp\_of} method was unit tested
with a mock callback function that also demonstrates how it could be
used with other material compositions.  The
\texttt{ndi\_atom\_comp\_of} and \texttt{ndi\_mass\_comp\_of}
functions were level tested by comparing the use case~3 (See Appendix~I)
output, using twenty-six element and four nuclide names from the
multigroup \texttt{lanl06ex} library, against the calculations of
\texttt{ndi\_zaid.py} for both the \texttt{lanl2006} and
\texttt{mt71x} continuous energy libraries.
  
\section{Conclusion}

The 1.1.0 release of \ndatk\ provides a solution to the element match
problem by resolving element names into the elemental or isotopic
tables available in the selected continuous energy \ndatk\ library
consistent with the calculated elemental tables in the corresponding
multigroup \NDI\ library.  The \texttt{Chart::atom\_comp\_of} and
\texttt{Chart::mass\_comp\_of} methods and
\texttt{ndi\_atom\_comp\_of} and \texttt{ndi\_mass\_comp\_of}
functions provide complete information about what is present in the
library versus what is present naturally.

Expansion of continuous energy elements but not multigroup elementals
satisfies legacy inputs for both multigroup (via elemental tables) and
continuous energy (via element expansion), but does not handle input
consistently between the two, does not faithfully reflect evaluations
for multigroup tables, and does not support named material
specifications in input.

However, as long as the input is restricted to only those tables
available in multigroup libraries and the continuous and multigroup
energy libraries are based on the same evaluation, this option
provides the desired seamless match between the underlying data sets
for two types of transport.

The option~3 use of the \texttt{Chart::atom\_comp\_of} and
\texttt{Chart::mass\_comp\_of} methods as callback functions in the
\texttt{Library::comp\_of} method suggest a path to some future
implementation of option~4, with code support for the expansion of
both elements and named materials by both continuous energy and
multigroup users.
 
A future Option~4 implementation could include:
\[
h:X \rightarrow C+M; L \subset X,
\]
as a callback extension of the option~3 functionality, and:
\[ 
\{X, +:X\times X \rightarrow X, \cdot:\mathbb{R}\times X \rightarrow X \},
\]
a composition vector space with unit amounts of nuclides as basis,
similar to that used in the calculations in \texttt{ndi\_zaid.py},
managed by a \texttt{Chart}-like container for named mixtures filled
with \texttt{Chart}-like curated data.  These code additions for
option~4 are straightforward; the real work lies in the collation,
curation, and ongoing support needed to maintain a useful materials
composition database.

\newpage
\begin{thebibliography}{99}

\bibitem{gray09} Mark G. Gray, \emph{\NDI\ Elemental Data},
  X--1:09--25, January 29, 2009
  
\bibitem{gray14} Mark G. Gray, \emph{The Nuclear Data Access Tool
  Kit}, XCP--5:14--005, November 14, 2014

\bibitem{gray15} Mark G. Gray, \emph{Proposed Solution to the
    Elemental Match Problem}, XCP--5:16--006, November 19, 2016

\bibitem{lakos96} John Lakos, \emph{Large-Scale \texttt{C++} Software
    Design}, Addison-Wesley, 1996

\bibitem{wiki} \emph{Cyclomatic Complexity}, 
  https://en.wikipedia.org/wiki/Cyclomatic\_complexity  

\end{thebibliography}

\footnotesize
\newpage
\section{Appendix~I: \texttt{use3\_ndi\_comp\_of.cc}} 
\label{app:use3} 
\lstinputlisting{../../test/use3_ndi_comp_of.cc}

\end{document}

