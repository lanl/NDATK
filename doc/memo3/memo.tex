% Template larmemos -- 13 Jan 99

\documentclass[12pt]{lamemo}
\usepackage{times}
\usepackage{amssymb}
\usepackage{listings}
\usepackage{graphicx}
\usepackage[all,cmtip]{xy}

\DeclareGraphicsExtensions{.jpg,.pdf,.mps,.png}
\DeclareGraphicsRule{*}{mps}{*}{}

\usepackage{color}
\definecolor{amber}{rgb}{1,.5,0}
\usepackage[normalem]{ulem}

%\usepackage{newcent}
%\pagestyle{secret}             % use this pagestyle for SRD
%\pagestyle{unclassified}       % use this pagestyle for U
%\pagestyle{official}           % use this pagestyle for OUO
\pagestyle{unmarked}           % use this pagestyle for no marking
% Fixed information
% All lines are required.
\divisionname{Computational Physics}	% center, project, or divison name 
\groupname{XCP--5}		% organization number and/or name
\phone{7-5341, 5-2879}		% sender phone, FAX number
\fromms{Mark G. Gray, F663}	% sender initials, mail stop
\originator{mgg}		% memo originator
\typist{mgg}			% memo typist

% Information on this memo
% The \toms, \refno, and \subject commands are required.

\toms{Christopher J. Werner, F663} % recipient initials, mail stop

\refno{XCP--5:16--006(U)} % reference number 

\subject{Proposed Solution to the Elemental Match Problem}

\date{November 19, 2015}

% Optional information:
\adc{dkp}
% \thru{nobody, MSnowhere}       % person(s) to send memo through
% \cy{aaa\\bbb}                  % copy list
\distribution{Edward D. Dendy, B218\\
Robert C. Ward, T086\\
Joann M. Campbell, T087\\
Kevin G. Honnell, F663\\
Terry R. Adams, F605\\
Steven D. Nolen, F605\\
Jeremy E. Sweezy, A143\\
Clell J. Solomon, F663\\
XCP--5 File}    % distribution list
% \enc{aaa\\bbb}                 % list of enclosures
% \encas                         % Enc. a/s 
% \attachments{aaa\\bbb}         % list of attachments
% \attachmentas                  % attachment as stated
% \attachmentsas                 % attachments as stated
% Text of the memo

\newcommand{\ndatk}{\texttt{ndatk}}
\newcommand{\MCATK}{\texttt{MCATK}}
\newcommand{\NDI}{\texttt{NDI}}
\newcommand{\zaid}{\texttt{SZAID}}
\newcommand{\MCNP}{\texttt{MCNP}}
\newcommand{\mcm}{\textcolor{green}{\checkmark}}
\newcommand{\mxm}{\textcolor{red}{$\times$}}
\begin{document}
\lstset{language=C++}

\maketitle			% make memo header

\section{Summary}
% I want to tell Chris that...

Production acceptance of the Monte Carlo Application Tool Kit (\MCATK)
requires continuous energy neutron transport data grouped to match
multigroup neutron transport libraries.  The Nuclear Data Access Tool
Kit (\ndatk) matches evaluated multigroup nuclides and elementals, but
cannot match calculated multigroup elementals.  I propose additions to
\ndatk\ that expand composition specifications by concentration of
nuclides, elementals, chemical substances, and named materials into
lists of nulcide concentrations.  Use of this tool would provide
better physical fidelity for both Monte Carlo and deterministic
transport simulations and fulfill the promise of \MCATK.

\section{Background}

Users of multigroup libraries specify material compositions as
concentrations of nuclides and elementals.  Although only nuclide
tables provide consistent, exact reactions, users were forced to use
elemental tables in older libraries because that was all old evaluations
provided.  For example, the \texttt{ENDF/B-VI} evaluation contains
elemental only tables for eighteen elements.

To support legacy input and user convenience, modern multigroup
libraries are augmented with calculated elemental tables.  Although
the \texttt{ENDF/B-VII} evaluation has only one elemental table, and
despite elemental table failures to provide consistent, exact
reactions, the \texttt{mendf71x} multigroup library provides
twenty-five additional elemental tables\cite{lee14} at user request.

This poses a problem for the comparison of continuous energy and
multigroup data based simulations.  \ndatk\cite{gray14} was written to
provide \MCATK\cite{adams14} users with the ability to match
multigroup library evaluated tables, but it cannot match calculated
multigroup elementals.

This memo examines various solutions to the elemental matching
problem and proposes additional features that could be added to
\ndatk\ to help solve it.

\newpage

\section{Discussion}

To truly meet the users needs for consistent, accurate, comparable
calculations between Monte Carlo and deterministic transport,
continuous energy (CE) and multigroup (MG) data libraries and/or the
code libraries which serve that data need to be augmented with
capabilities that meet the following criteria:
\begin{description}
  \item[Consistent Data: ] essentially match processed CE data sets to
    processed MG data sets
  \item[Evaluation True: ] faithfully match all processed CE and MG
    data sets to evaluated data
  \item[\textcolor{blue}{Legacy Input}: ] support ``dusty deck'' input
    specifications which include elementals
  \item[\textcolor{red}{Extended Input}: ] support extended input
    specifications which include chemical substances and named
    materials
\end{description}

Here are five possible solutions to the calculated elemental match
problem evaluated against these criteria:

\subsection{Option 0: Do Nothing}\mbox{}\\

Current continuous energy and multigroup data served by \ndatk\ and
\NDI, respectively, match processed data tables for only that subset
of data in evaluations (no calculated CE elementals), do not
faithfully match evaluated data (calculated MG elementals), match
``dusty decks'' only for multigroup data (no calculated CE
elementals), and do not support chemical substances or named
materials.  The diagram in Figure~\ref{fig:0} summarizes the criteria
relations among data and input for this option.

\setcounter{figure}{-1}
\begin{figure}\centering
\[
\xymatrix{
  & X \ar[ddl]_{\textcolor{red}{u}} \ar[ddr]^{\textcolor{red}{u}} & \\
  & L \ar[dl]^{\textcolor{blue}{u}} \ar@{<=>}[dr] \ar[u]_{\subset} & \\
  C \ar[rr]^{\subset} & & M \\
  & E \ar@{<=>}[ul] \ar[ur]_{\textcolor{blue}{d}} &
}
\]
\caption{Relations among Evaluations (E), Continuous Energy Libraries
  (C), Multigroup Libraries (M), Legacy Input (L), and Extended Input
  (X) in Option~0.  Continuous energy libraries are a subset of
  multigroup libraries, only continuous energy data faithfully matches
  evaluated data, the \textcolor{blue}{data team} calculates elemental
  tables for multigroup data to match legacy input while
  \textcolor{blue}{users} map legacy input into continuous energy
  data, and \textcolor{red}{users} map extended input into continuous
  energy and multigroup libraries.} \label{fig:0}
\end{figure}

\subsection{Option 1: Nuclides Only Input Specifications}\mbox{}\\

The simplest option to ensure a common set of evaluated data for
continuous energy and multigroup neutron transport is to restrict code
input to evaluated nuclide and elemental tables only.  The
\texttt{lanl2006} continuous energy\cite{gray14} and
multigroup\cite{white07} libraries already abide by this policy, and
it and could be further enforced by producing continuous energy and
multigroup libraries based solely on the \texttt{ENDF/B-VII}
evaluation, i.e., proper subsets of the current \texttt{mendf71x} and
\texttt{mt71x}\cite{conlin15} multigroup libraries.  This option
requires no code changes, but shifts the onus on users to manually
expand their material specifications in terms only of available
evaluation tables.  It uses only evaluated data consistently across
multigroup and continuous energy sets with full fidelity, but supports
neither legacy inputs nor named material specifications.  The diagram
in Figure~\ref{fig:1} summarizes the criteria relations among data and
input for this option.

\begin{figure}\centering
\[
\xymatrix{ & X \ar[ddl]_{\textcolor{red}{u}}
  \ar[ddr]^{\textcolor{red}{u}} & \\
  & L \ar[dl]^{\textcolor{blue}{u}} \ar[dr]_{\textcolor{blue}{u}}
  \ar[u]_{\subset} & \\
  C \ar@{<=>}[rr] & & M \\ & E \ar@{<=>}[ul] \ar@{<=>}[ur]& }
\]
\caption{Relations among Evaluations (E), Continuous Energy Libraries
  (C), Multigroup Libraries (M), Legacy Input (L), and Extended Input
  (X) in Option~1.  Continuous energy libraries essentially match
  multigroup libraries, both faithfully match evaluated data,
  \textcolor{blue}{users} map legacy input into continuous energy and
  multigroup libraries, and \textcolor{red}{users} map extended input
  into continuous energy and multigroup libraries.} \label{fig:1}
\end{figure}

Suggesting users stick with evaluation only input specifications
probably won't work; users not only demand backwards compatibility and
prefer to specify compositions in terms of elementals, but
additionally want to specify compositions in terms of chemical
substances, e.g., $\mathrm{H}_2\mathrm{O}$, and named mixtures, e.g.,
stainless steel\footnote{The more than 150 different grades of
  stainless steel makes this requirement particularly challenging.}.

\subsection{Option 2: Continuous Energy Elemental Tables}\mbox{}\\

Creating continuous energy elementals to match calculated multigroup
elementals also requires no host code work, but it would require major
data team work.  Whereas multigroup elemental cross sections are
trivially calculated by natural abundance weighted averages of
isotopic cross section vectors on a single fixed, common, energy
grid\cite[Kinematic Quantities]{gray09}, continuous energy elemental
cross sections would require calculating averages of functions defined
by piece-wise basis functions on several adaptive, unique, energy
grids.  The continuous energy results must then be expressed in terms
of basis functions on some appropriate resultant adaptive grid.
Because this calculation has never been part of any production work
done by the data team, it would be, at best, a research project
involving algorithms traditionally part of \texttt{NJOY} and hence
normally the province of T--2.

Nor could such a calculation be made fully consistent; elemental
``reactions'' cannot produce consistent products and do not conserve
energy in either multigroup or continuous energy form.  The average
reaction energy ($Q$) relies on an estimate of the neutron flux, which
is conveniently provided for multigroup elementals by the weight
function\cite[Kinetic Quantities]{gray09}; no such estimate exists for
continuous energy which is, of course, precisely its benefit.
Elementals are inherently less precise than nuclide tables, and a
principal reason to use continuous energy data is for the greater
precision it can offer.  The diagram in Figure~\ref{fig:2} summarizes
the criteria relations among data and input for this option.

\begin{figure}\centering
\[
\xymatrix{
  & X \ar[ddl]_{\textcolor{red}{u}} \ar[ddr]^{\textcolor{red}{u}} & \\
  & L \ar@{<=>}[dl] \ar@{<=>}[dr] \ar[u]_{\subset} & \\
  C \ar@{<=>}[rr] & & M \\
  & E \ar[ul]^{\textcolor{blue}{d}} \ar[ur]_{\textcolor{blue}{d}}&
}
\]
\caption{Relations among Evaluations (E), Continuous Energy Libraries
  (C), Multigroup Libraries (M), Legacy Input (L), and Extended Input
  (X) in Option~2.  Continuous energy libraries essentially match
  multigroup libraries, neither faithfully matches evaluated data, the
  \textcolor{blue}{data team} calculates elemental tables for
  continuous energy and multigroup libraries to match legacy input,
  and \textcolor{red}{users} map extended input into continuous energy
  and multigroup libraries.} \label{fig:2}
\end{figure}

This option would make continuous energy and multigroup elemental
treatments consistent, assuming an appropriate method to calculate
continuous energy elemental tables can be found, and it would support
legacy inputs, but it would fail faithfully reflect the underlying
evaluations, and would do nothing to support named materials in input.

\subsection{Option~3: Continuous Energy Elementals Expansion}\mbox{}\\ 

\begin{figure}\centering
\[
\xymatrix{
  & X \ar[ddl]_{\textcolor{red}{u}} \ar[ddr]^{\textcolor{red}{u}} & \\
  & L \ar[dl]^{\textcolor{blue}{h}} \ar@{<=>}[dr] \ar[u]_{\subset} & \\
  C \ar[rr]^{\subset} & & M \\
  & E \ar@{<=>}[ul] \ar[ur]_{\textcolor{blue}{d}} &
}
\]
\caption{Relations among Evaluations (E), Continuous Energy Libraries
  (C), Multigroup Libraries (L), Legacy Input (L), and Extended Input
  (X) in Option~3.  Continuous energy libraries are a subset of
  multigroup libraries, only continuous energy data faithfully matches
  evaluated data, the \textcolor{blue}{data team} calculates elemental
  tables for multigroup data to support legacy input while
  \textcolor{blue}{host codes} map legacy input into continuous energy
  data, and \textcolor{red}{users} map extended input into continuous
  energy and multigroup libraries.} \label{fig:3}
\end{figure}

The \texttt{Chart} class in the current production release of
\ndatk\ provides natural abundance data of isotopes by element as
illustrated in its release memo\cite[Appendix II]{gray14}.  This
capability could be used by host codes to expand an input
specification in terms of elementals and nuclides into one in terms of
nuclides alone.  For \texttt{C++} hosts minimal interface work is
needed, however codes in other languages would have to write a
cross-language interface to use this capability.  For this reason,
this option is unlikely to be embraced by multigroup codes, who would
likely continue to use the element tables available to them.
Continuous energy codes would then have to settle for approximate
comparisons where elemental tables are concerned.  Further, each code
that did adopt this solution would have to create its own custom
calculation code, defeating the purpose of having a common library.
The diagram in Figure~\ref{fig:3} summarizes the criteria relations
among data and input for this option.

Expansion of continuous energy elementals but not multigroup
elementals would satisfy legacy inputs for both multigroup (via
elemental tables) and continuous energy (via elemental expansion), but
would not handle input consistently between the two, would not
faithfully reflect evaluations for multigroup tables, and would not
support named material specifications in input.

\subsection{Option~4: Universal Elementals Expansion}\mbox{}\\

An \ndatk\ Version 1.1 release would add a flat interface to support
the use of Option~3 everywhere.  More work would be required for
\ndatk, but the availability of library routines in several languages
that provide common, curated, natural abundance data makes it
significantly more likely that several codes would adopt this
solution.

An \ndatk\ Version 2.0 release would add:
\begin{enumerate}
\item Polyvalent-program Pattern\cite[p. 281]{raymond03}\\
  The \ndatk\ API currently supports only \texttt{C++}; it is not as
  useful as it could be.  The API should be flattened to support
  \texttt{C} and \texttt{FORTRAN} linkages.  It should be augmented
  with command line and scripting interfaces.

\item Policy-based Class Design\cite[Ch. 1]{alexandrescu01}\\
  The \ndatk\ classes are not as flexible as they should be for a tool
  kit.  User selectable policies should be added to:
  \begin{itemize}
  \item specify behavior for missing \texttt{SZA} in a library
  \item specify behavior for incomplete \texttt{SZAID} match
  \item{specify temperature match behavior}
  \end{itemize}

\item Curated Physical Data Class with Unit Conversions\\
  A complete set of physical constants available with uncertainties in
  user specified units.
  
\item Curated Materials Data Class with Composition Calculations\\ A
  library of named materials defined by nuclide composition, together
  with a calculator for the nuclide composition of pure substances by
  chemical formula and the nuclide composition of elementals.

\end{enumerate}

Features~1 and 4 could solve the fundamental elemental matching
problem and hopefully attract widespread adoption by providing the
added benefit of a uniform, consistent, and convenient calculation of
material composition specification by mixture, substance, elemental,
and nuclide.

Additionally, Features~2 and 3 would improve the functionality
available to \ndatk's existing code base, and could be of use
elsewhere.  The augmentation with command line and scripting
interfaces could make \ndatk\ useful to legacy codes which would
benefit from preprocessing input without rewriting code, essentially
enabling Option~1 for those codes.  The diagram in Figure~\ref{fig:4}
summarizes the criteria relations among data and input for this
option.

\begin{figure}\centering
\[
\xymatrix{
  & X \ar[ddl]_{\textcolor{red}{h}} \ar[ddr]^{\textcolor{red}{h}} & \\
  & L \ar[dl]^{\textcolor{blue}{h}} \ar[dr]_{\textcolor{blue}{h}}
  \ar[u]_{\subset} & \\ 
  C \ar@{<=>}[rr] & & M \\
  & E \ar@{<=>}[ul] \ar@{<=>}[ur] &
}
\]
\caption{Relations among Evaluations (E), Continuous Energy libraries
  (C), Multigroup Libraries (M), Legacy Input (L), and Extended Input
  (X) in Option~4.  Continuous energy libraries essentially match
  multigroup libraries, both faithfully match evaluated data,
  \textcolor{blue}{host codes} map legacy input into continuous energy
  and multigroup libraries, and \textcolor{red}{host codes} map
  extended input into continuous energy and multigroup
  libraries. } \label{fig:4}
\end{figure}

This option satisfies all criteria.

\section{Conclusion}

Production acceptance and widespread use of \MCATK\ requires both
\ndatk, to match evaluated nuclide and elemental tables in libraries,
and a solution to the calculated multigroup elemental matching
problem.  The solutions considered to the matching problem are
summarized in Table~\ref{tbl:summary}.

\begin{table}\centering
  \caption{Evaluation of Five Solution Options Versus
    Criteria.}\label{tbl:summary}
\begin{tabular}{l|cccc} 
  & \multicolumn{4}{c}{Criteria} \\ \hline
  Options & Consistent Data & Evaluation True &
  \textcolor{blue}{Legacy Input} & \textcolor{red}{Extended Input} \\ \hline
  0) Do Nothing          & \mxm & \mxm & \mxm & \mxm \\
  1) Use Nuclides Only   & \mcm & \mcm & \mxm & \mxm \\
  2) Make CE Elementals  & \mcm & \mxm & \mcm & \mxm \\
  3) Do CE Expansion     & \mxm & \mxm & \mcm & \mxm \\
  4) Universal Expansion & \mcm & \mcm & \mcm & \mcm
\end{tabular}
\end{table}
Each solution requires a different mix of work from the nuclear data
team, host codes, and end users: Options~1 and 2 require major work by
end users, Option~2 requires major work by the nuclear data team,
Option~3 requires major work by \MCATK\ and other Monte Carlo codes,
while Option~4 shares the work between \ndatk\ and host codes.  Only
Option~4 adds additional features through the extension of \ndatk.
The new features let \MCATK, multigroup transport codes, host codes,
and users expand input specifications containing nuclides, elementals,
chemical substances and named materials into nuclide concentration
specifications.  Based on the estimates used for \ndatk\ 1.0.0, I
estimate the effort involved in Option~4 at 3 person-months or 1/4 FTE
for 12 months.  I recommend Option~4.

\newpage
\begin{thebibliography}{99}

\bibitem{lee14} Mary Beth Lee and Mark G. Gray, \emph{Updated
  \NDI\ Library \texttt{MENDF71X} and its Production-Depletion Chain
  Data Sets}, XCP--5:14--013, June 19, 2014
  
\bibitem{gray14} Mark G. Gray, \emph{The Nuclear Data Access Tool
  Kit}, XCP--5:14--005, November 14, 2014

\bibitem{adams14} T. Adams et al., \emph{Monte Carlo Application
  ToolKit (MCATK)}, Ann. Nucl. Energy (2014),
  http://dx.doi.org/10.1016/j.anucene.2014.08.047

\bibitem{white07} Morgan C. White, \emph{Release of the
  \texttt{LANL2006} \NDI\ Neutron Multigroup Data Library},
  X--1:MCW--07--67, July 20, 2007
  
\bibitem{conlin15} Jeremy Lloyd Conlin et al., \emph{\texttt{MT71X}
  Multi-temperature Library Based on \texttt{ENDF/B-VII.1}},
  XCP--5:15--048, October 29, 2015

\bibitem{gray09} Mark G. Gray, \emph{NDI Elemental Data},
  X--1:09--25, January 29, 2009
  
\bibitem{raymond03} Eric S. Raymond, \emph{The Art of \texttt{UNIX}
  Programming}, Addison-Wesley, 2003

\bibitem{alexandrescu01} Andrei Alexandrescu, \emph{Modern
  \texttt{C++} Design}, Addison-Wesley, 2001

\end{thebibliography}

\end{document}

