\documentclass[clock]{slides}

%\onlyslides{1-100}

\ifx\pdftexversion\undefined
\usepackage[dvips]{graphicx}
\else
\usepackage[pdftex]{graphicx}
\DeclareGraphicsExtensions{.jpg,.pdf,.mps,.png}
\DeclareGraphicsRule{*}{mps}{*}{}
\fi
\usepackage{color}
\usepackage[normalem]{ulem}
                                                                               
\setlength{\topskip}{0cm}
\newenvironment{xslide}[1][logo.jpg]{\begin{slide} \tiny
\textcolor{blue}{\underline{ndatk}} \hfill
\includegraphics[height=36pt]{#1}
\normalsize}{\vfill\tiny
\textcolor{blue}{\hrulefill \\LANL XCP--5}
\end{slide}}
                                                                               
%%%%%%%%%%%%%%%%%%%%%%%%%%%%%%%%%%%%%%%%%%%%%%%%%%%%%%%%%%%%%%%%%%%%%%%%
                              
\newcommand{\ndatk}{\texttt{ndatk}}
\newcommand{\MCATK}{\texttt{MCATK}}
\newcommand{\NDI}{\texttt{NDI}}
\newcommand{\MCNP}{\texttt{MCNP}}
\newcommand{\zaid}{\texttt{SZAID}}
                                                 
\begin{document}

\begin{xslide}

\begin{center}\Large
Nuclear Data Access Tool Kit\\
Wednesday 7 October 2015\\
\vspace{2in}
Mark G. Gray\\ 
\end{center}

\end{xslide}

\addtime{60}
\begin{note}\small
The Nuclear Data Access Tool Kit (\ndatk) is a C++ code library that
provides access to continuous energy neutron cross section table
meta-data traditionally found in the cross section directory file
(\texttt{XSDIR}), grouping nuclide tables into data libraries
consistent with the multi-group neutron cross section data provided by
the Nuclear Data Interface (\NDI).

This talk, which was originally presented in part at a \MCATK\ demo,
1300 Monday 20 May 2013, overviews the design and interface of ndatk
and updates its status.\footnote{Thanks to Terry Adams for review and
  comments 10/15/13}

This talk was subsequently updated for presentation to XCP--3
Wednesday 7 October 2015.
\end{note}

\begin{xslide}
\begin{center}\Large
Nuclear Data Supply Chain
\end{center}

\begin{center}
\includegraphics[height=5in]{nds.pdf}
\end{center}
\textcolor{green}{You Are Here!}
\end{xslide}

\addtime{240}
\begin{note}\small
Nuclear data is:
\begin{itemize}
\item \textcolor{red}{Produced by models backed by experiment}
\item Collated, Processed, Verified, Validated and Packaged for use.
\item \textcolor{blue}{Used in simulation}
\end{itemize}

The Nuclear data team:
\begin{itemize}
\item Manages some of the \textcolor{red}{experiments and modeling
  that produces ND}
\item Processes continuous and multigroup data including V\&V
\item Develops and maintains code to access MG and
  \textcolor{green}{CE} data
\item Supports user \textcolor{blue}{code development and data use}
\end{itemize}

You are here: \textcolor{green}{\ndatk} is code to access CE
meta-data.
\end{note}

\begin{xslide}
\begin{center}\large
\ndatk\ Requirements
\end{center}

\begin{enumerate}
\item MG szaid $\Leftrightarrow$ CE szaid
\item Polyvalent program access pattern
\item \NDI\ multi-temperature features
\item \NDI\ \texttt{SZA} translation features
\item Curated Data
\item Continuous Energy Libraries
\end{enumerate}
\end{xslide}

\addtime{120}
\begin{note}\small

On Monday 10 December 2012 various potential stakeholders of ndatk met
to discuss requirements.  They suggested the following set:  
\begin{enumerate}
\item Original requirement from scf, rcl
\item Original target \texttt{C++} library for \texttt{MCATK}; flat
  \texttt{C}, \texttt{FORTRAN} library;
  scripting interface likely needed for others
\item \NDI\ multi-temperature features; how does this work with reqt. 1?
\item \NDI\ \texttt{SZA} translation features; how does this work with
  reqt. 1?
\item Curated Data
  \begin{itemize}
    \item Documented sources
    \item Documented transformations
    \item Public release history
  \end{itemize}
\end{enumerate}

On Wednesday 23 January 2013 the data team met with management for
authorization to proceed.  One requirement, which resolved conflicts
among previous requirements was added:
\begin{enumerate}
\setcounter{enumi}{5}
\item Continuous Energy Libraries
\end{enumerate}

\end{note}

\begin{xslide}
\begin{center}\Large
\ndatk\ Constraints
\end{center}

\begin{quote}
Things are the way they are because they got that way.
\flushright -- Kenneth Boulding
\end{quote}
\begin{quote}
If you don't think too good, don't think too much.
\flushright -- Ted Williams
\end{quote}
\begin{quote}
I made this very long because I did not have the leisure to make it
shorter.
\flushright -- Blaise Pascal
\end{quote}
\end{xslide}

\addtime{120}
\begin{note}\small

In addition to the explicit stakeholder requirements, additional
requirements come implicitly from stakeholder expectations, my
abilities, and resource constraints, namely:
\begin{description}
\item[Boulding's backward basis: ] The \texttt{XSDIR} and \NDI\ work
  the way they currently do because they fulfill certain requirements
  for their users.  Any solution I create here is constrained by their
  user's expectations.
\item[William's limit: ] Since I don't ``code too good'', I
  intentionally want to design the smallest number of lines of code
  that will do the job.
\item[Pascal's paradox: ] Unfortunately I probably don't have the
  leisure to achieve my William's limit.
\end{description}

Because of these constraints I decided to:
\begin{itemize}
\item analyze to consolidate, prioritize requirements,
\item design from existing code and data,
\item prototype extensively using an interpreted language
\item prioritize functionality and stage delivery,
\end{itemize}
as a means to both manage the schedule and mitigate the risks.

\end{note}

\begin{xslide}
\begin{center}\Large
\texttt{XSDIR} Deconstructed: Data
\end{center}
The Data Directory File (\texttt{XSDIR}) has three sections:
\begin{enumerate}
\item DATAPATH\\
  ``The directory where the data libraries (sic) are stored...''
\item ATOMIC WEIGHT RATIOS\\
  ``... free format pairs of ZAID (sic) AWR...''
\item DIRECTORY\\
``...listing of available data tables.''
\end{enumerate}
\end{xslide}

\addtime{120}
\begin{note}\small

Applying Boulding's backward basis to the \texttt{XSDIR}:
\begin{enumerate}
\item The DATAPATH section in \texttt{XSDIR} is a small part of the
  policy that \texttt{MCNP} uses to find its data.  The ndatk should
  implement a full file search policy, including default (\NDI\ like),
  API, and environment specified input paths.
\item The AWR section in the \texttt{XSDIR} is simultaneously
  redundant with the atomic weight ratios provided both in the
  DIRECTORY section and on the tables, and insufficient for nuclide
  scalar data needed.  The ndatk needs to include it and natural
  abundances from a identified, standard source, to calculate isotopic
  loadings from elemental input specifications.
\item The DIRECTORY section in the \texttt{XSDIR} provides mapping
  from unique table identifiers (\zaid s) to file meta-data, and a
  preferred order for identifiers which begin with the same \texttt{SZA}.  The
  ndatk needs to provide the same data, but additionally grouped by
  library subsets.
\end{enumerate}

\end{note}

\begin{xslide}
\begin{center}\Large
\NDI\ Deconstructed: API
\end{center}
\begin{itemize}
\item \verb+ndi2_get_TYPE_DIM(handle, KEY)+
\item \verb+ndi2_get_TYPE_DIM_n(handle, KEY, index)+
\item \verb+ndi2_get_TYPE_DIM_x(handle, KEY, name)+
\end{itemize}
where
\begin{itemize}
\item \verb+TYPE = int|float|string+
\item \verb+DIM = val|vec+
\end{itemize}
\end{xslide}

\addtime{60}
\begin{note}\small

Applying Boulding's backward basis to the \NDI:
\begin{description}
\item[ndi2: ] namespace
\item[get: ] data query
\item[TYPE: ] return data type  
\item[DIM: ] return data extent
\item[n or x: ] additional argument indicator
\item[KEY: ] keyword data selector
\end{description}

3 types * 2 extents *3 argument indicator * number of keys =
18 functions * number of keys

\end{note}
 
\begin{xslide}
\begin{center}\Large
\ndatk\ Problem Frame
\end{center}

\begin{center}
\includegraphics[height=4in, width=5in]{ndatk.10}
\end{center}

\small
\begin{description}
\item[Requirement: ] $\mathrm{R} \Rightarrow (\mathrm{UD} \sim
  \mathrm{ND}) \wedge \mathrm{status}$
\item[Specification: ] $\mathrm{UC!R} \times \mathrm{ndatk!ND}
  \rightarrow \mathrm{ndatk!UD}$
\end{description}

\end{xslide}

\addtime{240}
\begin{note}\small
A Commanded Information Problem Frame.  LHS (dashed lines) of the
diagram labels requirements on user code requests (R), user data
formats (UD), nuclear data formats (ND), and the primary requirement:
In response to a request R make the user's data (UD) correspond to the
nuclear data (ND) and return a status indicator of the action.

The RHS (solid lines) of the diagram labels specifications of user
calls (UC!R), data reads (ndatk!ND), and API return data (ndatk!UD).

The ndatk package is a levelized logical design, consisting of an
application programming interface (API), a finite state machine (FSM),
data file readers, and utility code.
\begin{itemize}
\item The API layer insulates users from changes in the data; new data
  types can be added without requiring interface rewrites
\item The FSM and utilities layers add value; through them the interface
  can transform file data or calculate missing data.
\end{itemize}
It is this design that successfully insulates \NDI\ code users from \NDI\
data changes.
\end{note}

\begin{xslide}
\begin{center}\Large
\ndatk\ Package Diagram
\end{center}
\begin{center}
\includegraphics[height=5in]{package.pdf}
\end{center}
\end{xslide}

\addtime{180}
\begin{note}\small
The \ndatk\ package is a three level physical design:
\begin{enumerate}
\setcounter{enumi}{-1}
\item \texttt{constants} and \texttt{utils} modules and the
  \texttt{CuratedData} and \texttt{Finder} classes.
\item \texttt{translate\_isomer} module and \texttt{Exsdir} class
\item \texttt{Library} and \texttt{Chart} classes
\end{enumerate}

The \ndatk\ namespace contains: 
\begin{itemize}
\item Three support modules:
\begin{description}
\item[constants: ] Nuclear data physical constants
\item[utils: ] STL-11, STL-14, Boost, Loki functions
\item[translate\_isomer: ] Nuclide name translation
\end{description}
\item Five classes:
\begin{description}
\item[CuratedData: ] Data name, type, provenance
\item[Finder: ] Search \texttt{PATH} list for files
\item[Exsdir: ] Table meta-data by \zaid
\item[Chart: ] Chart of the nuclides data
\item[Library: ] Subset of \zaid s corresponding to MG Libraries
\end{description}
\end{itemize} 
and depends only on the STL-03 namespace.
\end{note}

\begin{xslide}
\begin{center}\Large
\ndatk\ Class Diagram
\end{center}
\begin{center}
\includegraphics[height=5in]{class.pdf}
\end{center}
\end{xslide}

\addtime{180}
\begin{note}\small
The \texttt{XSDIR} features in \ndatk\ divide its functionality into
the Chart (ATOMIC WEIGHT RATIOS) and Exsdir (DIRECTORY) classes; add a
Library class to encapsulate the grouping \zaid s into libraries, and
CuratedData class to encapsulate data provenance.

This class diagram shows only the important public assessor methods
and internal state for the use cases to come.

Talk about how these classes work to meet requirements.

The \NDI\ feature in ndatk are:
\begin{description}
\item[ndatk: ] namespace implied by construction
\item[const: ] data query implied by signature
\item[TYPE: ] return type implied by signature
\item[DIM: ] return extent implied by signature
\item[additional arguments: ] implied by signature
\item[KEY: ] data selector inherent in function name
\end{description}
\texttt{C++}'s strong static type checking enforces correct use at
compile time.
\end{note}

\begin{xslide}
\begin{center}\Large
\ndatk\ \textcolor{red}{Exsdir}
\end{center}
\small
\begin{verbatim}
type: ndatk_exsdir_1.0
name: exsdir
provenance:
Provenance header added by Mark G. Gray on 2014-08-26.
%%
provenance:
Added mcplib84 data by Mark G. Gray on 2015-03-13.
%%
provenance:
Added mendf71x data by Mark G. Gray on 2015-05-18.
%%
DIRECTORY
chart 0.0 endf71x/chart
lanl2006 0.0 lanl2006/library
endf71x 0.0 endf71x/library
mtmg08 0.0 mtmg08/library
mcplib84 0.0 mcplib84/mcplib84.lib
e68g_103 0.0 mcplib84/e68g_103.lib
include lanl2006/xsdir
include endf71x/xsdir
include mtmg08/xsdir
include mcplib84/xsdir
include mendf71x/xsdir
\end{verbatim}
\end{xslide}

\addtime{60}
\begin{note}\small

The Exsdir data file is typical of \ndatk\ data:
\begin{itemize}
\item Record-Jar format
\item Type id
\item Name
\item History
\item Data
\end{itemize}

The DIRECTORY entries are containted in the included files.

\end{note}

\begin{xslide}
\begin{center}\Large
\texttt{mendf71x} \textcolor{blue}{Library}
\end{center}
\small
\begin{verbatim}
type: ndatk_library_1.0

name: mendf71x

provenance:
File originally extracted by Mark G. Gray from 
../endf71x/library on 2015-05-15.  Added As73 
and two fission products.
%%

ids:
# sza szaid
1001  1001.710nc
1002  1002.710nc
1003  1003.710nc
2003  2003.710nc
2004  2004.710nc
3006  3006.710nc
3007  3007.710nc
      .
      .
      .
\end{verbatim}
\end{xslide}

\addtime{60}
\begin{note}\small
Library's data is a multi-map from \texttt{SZA} to \zaid.
\end{note}

\begin{xslide}
\begin{center}\Large
\ndatk\ Code...
\end{center}\small
\begin{tabular}{lr|l}\hline
\texttt{ndatk} & Lines of Code & Language \\ \hline
Prototype Code & 6442 & \texttt{Python} \\ \hline
Library Code & 830 & \texttt{C++} \\
Unit Test Code & 316 & \texttt{C++} \\
Use Case Code & 131 & \texttt{C++} \\ \hline
Total Production Code & 1277 & \texttt{C++}
\end{tabular}

\medskip
\begin{center}\Large
and Data
\end{center}\tiny
\begin{tabular}{rrrr|rr}\hline
\texttt{ndatk} & Tables & Temp & Date & \texttt{NDI} & \texttt{MCNP} \\ \hline
\texttt{lanl2006} & 180 & 1 & 5/30/13 & $\approx$ \texttt{lanl2006} & \\
\texttt{endf71x} & 424 & 7 & 11/15/13 & & \texttt{endf71x} \\
\texttt{mtmg08} & 64 & 10 & 5/18/14 & $\subset$ \texttt{mtmg08} \\ \hline
\textcolor{blue}{\texttt{mcplib84}} & 100 & 1 & 3/13/15 &  & \texttt{mcplib84}\\
\textcolor{blue}{\texttt{e68g\_103}} & 523 & 1 & 3/13/15 & \texttt{e68g\_103} &\\ \hline
\textcolor{green}{\texttt{mendf71x}} & 427 & 1 & 5/18/15 &
\texttt{mendf71x} & \\ \hline
\textcolor{red}{\texttt{mt71x}} & 427 & 23 & TBA & \texttt{mt71x} & 
\end{tabular}

\end{xslide}

\begin{note}\small
The production release of \ndatk\ contains a code library,
275,343 unit test and 33 use cases which were designed and prototyped
based on \texttt{Python} code.  The basic \texttt{COCOMO} model
predicts a 4 person-month effort for the total amout of production
code, which compares favorably to the actual 1/4 person time 17 month
effort.

The production release of \ndatk\ 1.0.0 contained three
continuous energy neutron transport libraries that correspond to the
indicated \NDI\ and \MCNP\ libraries.  Subsequent data
releases contained \textcolor{blue}{two} continuous energy photon
transport libraries, which correspond to the indicated \NDI\
and \MCNP\ libraries, and \textcolor{green}{a} continuous
energy neutron transport library which corresponds to the indicated
\NDI\ library.

A future release will add \textcolor{red}{a} continuous energy
multi-temperature neutron transport library that corresponds the the
indicated \NDI\ library.
\end{note}

\begin{xslide}
\begin{center}\Large
\ndatk\ Use Cases
\end{center}
\begin{enumerate}
\setcounter{enumi}{-1}
\item Single Temperature Library Access
\item Calculate Elemental Atomic Weight Ratios (AWR) from Isotopic
  Data
\item Multi-temperature Library Access
\end{enumerate}
\end{xslide}

\addtime{360}
\begin{note}\small
\begin{enumerate}
\setcounter{enumi}{-1}
\item Create \texttt{Exsdir} by name; create \texttt{Library} by name
  and \texttt{Exsdir}; retrieve isotope file name and line number
  offset by isomer name.

\item Create \texttt{Exsdir} by name; create \texttt{Chart} by name
  and \texttt{Exsdir}; calculate elemental atomic weight ratio from
  isotopic abundances and atomic weight ratios.

\item Create \texttt{Exsdir} by name; create \texttt{Library} by name
  and \texttt{Exsdir}; retrieve isotope file name and line number
  offset by isomer name; for each isomer temperature in library
  retrieve table name.
\end{enumerate}

\end{note}

\begin{xslide}
\begin{center}\Large
\texttt{use0\_Library.cc}
\end{center}
\tiny
\begin{verbatim}
  ndatk::Exsdir x(argv[1]);     // 1. Make Exsdir from filename
  cout << "Created Exsdir " << x.name()
       << " with " << x.number_of_tables() << " tables." << endl;

  ndatk::Library l(argv[2], x); // 2. Make Library from name, Exsdir
  cout << "Created Library " << l.name()
       << " with " << l.number_of_tables() << " tables." << endl;

  for (int i = 0; i < n; i++) { // 3. List some sza's info
    string s(sza[i]);
    cout << s;
    string szaid = l.table_identifier(s);
    if (szaid == "") 
      cout << "Couldn't find " << s << " in Library!" << endl;
    else
      cout << "(" << szaid <<"): " 
           << l.file_name() << "@" << l.address() << endl;
  }
\end{verbatim}
\end{xslide}

\addtime{60}
\begin{note}\small
\begin{itemize}
\item Create \texttt{Exsdir} by file name.  Uses \texttt{Finder} to
  associate file name with readable file of the right type.  Once
  \texttt{Exsdir} is constructed, it can be querried for all data typically
  found in \texttt{XSDIR}.
\item Create \texttt{Library} by name and \texttt{Exsdir}.  Uses
  \texttt{Exsdir} to map library name to file name and \texttt{Finder}
  to associate file name with readable file of the right type.  Once
  \texttt{Library} is constructed it can be querried for \zaid by
  \texttt{SZA} or symbolic name, and provide \texttt{XSDIR}
  information including the absolute file path.
\item Find a list of SZA values from \texttt{Library}.
\end{itemize}

\end{note}

\begin{xslide}
\begin{center}\Large
Status
\end{center}
\begin{enumerate}
\item \sout{MG szaid $\Leftrightarrow$ CE szaid}
\item \textcolor{red}{Polyvalent program pattern access}
\item \textcolor{green}{\NDI\ multi-temperature features}
\item \textcolor{green}{\NDI\ \texttt{SZA} translation features}
\item \textcolor{green}{Curated Data}
\item \textcolor{green}{Continuous Energy Libraries}
\end{enumerate}

\end{xslide}

\addtime{60}
\begin{note}\small
\begin{enumerate}
\item MG elementals make one-to-one translation impossible; item six
  supersedes this requirement
\item OO \texttt{C++} library API implemented; others may follow
\item Done
\item Done
\item Done
\item Done
\end{enumerate}
\end{note}

\begin{xslide}
\begin{center}\Large
To Do
\end{center}

\begin{itemize}
\item \texttt{SZA} and \zaid\ search policies in Library
\item Temperature match policy in \texttt{Library}
\item Flat \texttt{C}, \texttt{FORTRAN} interface (Requirement 2)
\item Command line interface (Requirement 2)
\item Physical Constants class
\end{itemize}
\end{xslide}

\addtime{60}
\begin{note}\small
\begin{itemize}
\item Library currently looks for \texttt{SZA} in library list only
  and partial or complete \zaid in XSDIR only; a half dozen other
  search policies are possible and could be implemented with
  appropriate templating, e.g. library search path (including Exsdir)
  or z* regex match (for photon data).
\item Library currently looks for nearest temperature; a half dozen
  other search policies are possible and could be implemented with
  appropriate templating.
\item OO \texttt{C++} API currently implemented; additional flat API
  could be defined and code written.
\item OO \texttt{C++} API currently implemented; stand alone CLI
  application could be written.
\item constants.hh define a subset of physical constants in ndatk
  appropriate units; could define curated Constants class with user
  defined units.
\end{itemize}
\end{note}
\end{document}
