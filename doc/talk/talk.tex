\documentclass[clock]{slides}

%\onlyslides{1-100}

\ifx\pdftexversion\undefined
\usepackage[dvips]{graphicx}
\else
\usepackage[pdftex]{graphicx}
\DeclareGraphicsExtensions{.jpg,.pdf,.mps,.png}
\DeclareGraphicsRule{*}{mps}{*}{}
\fi
\usepackage{color}
\usepackage[normalem]{ulem}
                                                                               
\setlength{\topskip}{0cm}
\newenvironment{xslide}[1][logo.jpg]{\begin{slide} \tiny
\textcolor{blue}{\underline{ndatk}} \hfill
\includegraphics[height=36pt]{#1}
\normalsize}{\vfill\tiny
\textcolor{blue}{\hrulefill \\LANL XCP--5}
\end{slide}}
                                                                               
%%%%%%%%%%%%%%%%%%%%%%%%%%%%%%%%%%%%%%%%%%%%%%%%%%%%%%%%%%%%%%%%%%%%%%%%
                                                                               
\begin{document}

\begin{xslide}

\begin{center}\Large
Nuclear Data Access Tool Kit\\
\today\\
\vspace{2in}
Mark G. Gray\\ 
\end{center}

\end{xslide}

\addtime{60}
\begin{note}\small
The Nuclear Data Access Tool Kit (ndatk) is a C++ code library that
provides access to continuous energy neutron cross section table
meta-data traditionally found in the cross section directory file
(xsdir), grouping nuclide tables into data libraries consistent with
the multi-group neutron cross section data provided by the Nuclear
Data Interface (NDI).

This talk, which was originally presented in part at a MCATK demo,
1300 Monday 20 May 2013, overviews the design and interface of ndatk
and updates its status.\footnote{Thanks to Terry Adams for review and
  comments 10/15/13}
\end{note}

\begin{xslide}
\begin{center}\large
NDATK Requirements
\end{center}

\begin{enumerate}
\item MG szaid $\Leftrightarrow$ CE szaid
\item Several user access through polyvalent program pattern
\item NDI multi-temperature features
\item NDI sza translation features
\item Curated Data
\item Continuous Energy Libraries
\end{enumerate}
\end{xslide}

\addtime{120}
\begin{note}\small
On Monday 10 December 2012 various potential stakeholders of ndatk met
to discuss requirements.  They suggested the following set:  
\begin{enumerate}
\item Original requirement from scf, rcl
\item Original target C++ library for MCATK; flat C/FORTRAN library;
  scripting interface likely needed for others
\item NDI multi-temperature features; how does this work with reqt. 1?
\item NDI sza translation features; how does this work with reqt. 1?
\item Curated Data
  \begin{itemize}
    \item Documented sources
    \item Documented transformations
    \item Public release history
  \end{itemize}
\end{enumerate}
On Wednesday 23 January 2013 the data team met with management for
authorization to proceed.  One requirement, which resolved conflicts
among previous requirements was added:
\begin{enumerate}
\setcounter{enumi}{5}
\item Continuous Energy Libraries
\end{enumerate}
\end{note}

\begin{xslide}
\begin{center}\Large
NDATK Constraints
\end{center}
\begin{quote}
Things are the way they are because they got that way.
\flushright -- Kenneth Boulding
\end{quote}
\begin{quote}
If you don't think too good, don't think too much.
\flushright -- Ted Williams
\end{quote}
\begin{quote}
I made this very long because I did not have the leisure to make it
shorter.
\flushright -- Blaise Pascal
\end{quote}
\end{xslide}

\addtime{120}
\begin{note}\small
In addition to the explicit stakeholder requirements, additional
requirements come implicitly from stakeholder expectations, my
abilities, and resource constraints, namely:
\begin{description}
\item[Boulding's backward basis: ] The xsdir and NDI work the way they
  currently do because they address certain requirements for their
  users.  Any solution I create here is constrained by their user's
  expectations. 
\item[William's limit: ] Since I don't ``code too good'', I
  intentionally want to design the smallest number of lines of code
  that will do the job.
\item[Pascal's paradox: ] Unfortunately I probably don't have the
  leisure to achieve my William's limit.
\end{description}
\end{note}

\begin{xslide}
\begin{center}\Large
XSDIR Deconstructed
\end{center}
The Data Directory File (XSDIR) has three sections:
\begin{enumerate}
\item DATAPATH\\
  ``The directory where the data libraries (sic) are stored...''
\item ATOMIC WEIGHT RATIOS\\
  ``... free format pairs of ZAID (sic) AWR...''
\item DIRECTORY\\
``...listing of available data tables.''
\end{enumerate}
\end{xslide}

\addtime{120}
\begin{note}\small
Applying Boulding's backward basis to the xsdir:
\begin{enumerate}
\item The DATAPATH section in XSDIR is a small part of the policy that
  MCNP uses to find its data.  The ndatk should implement a full file
  search policy, including default, API, and environment specified
  input paths.
\item The AWR section in the XSDIR is simultaneously redundant with
  the awr provided in the DIRECTORY section and on the tables, and
  insufficient for nuclide scalar data needed.  The ndatk needs to
  include it and natural abundances from a identified, standard
  source, to calculate isotopic loadings from elemental input
  specifications.
\item The DIRECTORY section in the XSDIR provides mapping from unique
  table identifiers (SZAIDs) to file meta-data, and a preferred order
  for identifiers which begin with the same SZA.  The ndatk needs to
  provide the same data, but grouped by library subsets.
\end{enumerate}
\end{note}

\begin{xslide}
\begin{center}\Large
NDI Deconstructed: API
\end{center}
\begin{itemize}
\item \verb+ndi2_get_TYPE_DIM(handle, KEY)+
\item \verb+ndi2_get_TYPE_DIM_n(handle, KEY, index)+
\item \verb+ndi2_get_TYPE_DIM_x(handle, KEY, name)+
\end{itemize}
where
\begin{itemize}
\item \verb+TYPE = int|float|string+
\item \verb+DIM = val|vec+
\end{itemize}
\end{xslide}

\addtime{60}
\begin{note}\small
Applying Boulding's backward basis to the NDI:
\begin{description}
\item[ndi2: ] namespace
\item[get: ] data query
\item[TYPE: ] return data type  
\item[DIM: ] return data extent
\item[n or x: ] additional argument indicator
\item[KEY: ] keyword data selector
\end{description}

3 types * 2 extents *3 argument indicator * number of keys =
18 functions * number of keys
\end{note}
 
\begin{xslide}
\begin{center}\Large
NDATK Class Diagram
\end{center}
\begin{center}
\includegraphics{class.pdf}
\end{center}
\end{xslide}

\addtime{180}
\begin{note}\small
The xsdir features in ndatk divide its functionality into the Chart
(ATOMIC WEIGHT RATIOS) and Exsdir (DIRECTORY) classes; add a Library
class to encapsulate the grouping SZAIDs into libraries, and
CuratedData class to encapsulate data provenance.

This class diagram shows only the important public assessor methods
and internal state for the use cases to come.

Talk about how these classes work to meet requirements.
\end{note}

\begin{xslide}
\begin{center}\Large
NDATK API
\end{center}

\begin{itemize}
\item \verb+object.KEY()+
\item \verb+object.KEY(index)+
\item \verb+object.KEY(name)+
\end{itemize}
\end{xslide}

\addtime{60}
\begin{note}\small
The NDI feature in ndatk are:
\begin{description}
\item[ndatk: ] namespace implied by construction
\item[const: ] data query implied by signature
\item[TYPE: ] return type implied by signature
\item[DIM: ] return extent implied by signature
\item[additional arguments: ] implied by signature
\item[KEY: ] data selector inherent in function name
\end{description}
C++'s strong static type checking enforces correct use at compile time. 
\end{note}    

\begin{xslide}
\begin{center}\Large
NDATK Problem Frame
\end{center}

\begin{center}
\includegraphics[height=5in, width=6in]{ndatk.10}
\end{center}

\end{xslide}

\addtime{240}
\begin{note}\small
A Commanded Information Problem Frame.  LHS (dashed lines) of the
diagram labels requirements on user code requests (R), user data
formats (UD), nuclear data formats (ND), and the primary requirement:
In response to a request R make the user's data (UD) correspond to the
nuclear data (ND) and return a status indicator of the action.

The RHS (solid lines) of the diagram labels specifications of user
calls (UC!R), data reads (ndatk!ND), and API return data (ndatk!UD).

The ndatk package is a levelized logical design, consisting of an
application programming interface (API), a finite state machine (FSM),
data file readers, and utility code.
\begin{itemize}
\item The API layer insulates users from changes in the data; new data
  types can be added without requiring interface rewrites
\item The FSM and utilities layers add value; through them the interface
  can transform file data or calculate missing data.
\end{itemize}
It is this design that successfully insulates NDI code users from NDI
data changes.
\end{note}

\begin{xslide}
\begin{center}\Large
NDATK Package Diagram
\end{center}
\begin{center}
\includegraphics[height=5in]{package.pdf}
\end{center}
\end{xslide}

\addtime{180}
\begin{note}\small
The ndatk package is a three level physical design:
\begin{enumerate}
\setcounter{enumi}{-1}
\item \texttt{constants} and \texttt{utils} modules and the
  \texttt{CuratedData} class.
\item \texttt{translate\_isomer} module and \texttt{Exsdir} class
\item \texttt{Library} and \texttt{Chart} classes
\end{enumerate}

The ndatk namespace contains: 
\begin{itemize}
\item Three support modules:
\begin{description}
\item[constants: ] Nuclear data physical constants
\item[utils: ] STL-11, STL-14, Boost, Loki functions
\item[translate\_isomer: ] Nuclide name translation
\end{description}
\item Four classes:
\begin{description}
\item[CuratedData: ] Data name, type, provenance
\item[Exsdir: ] Table meta-data by SZAID
\item[Chart: ] Chart of the nuclides data
\item[Library: ] Subset of SZAIDS corresponding to MG Libraries
\end{description}
\end{itemize} 
and depends only on the STL-03 namespace.
\end{note}

\begin{xslide}
\begin{center}\Large
NDATK Use Cases
\end{center}
\begin{enumerate}
\setcounter{enumi}{-1}
\item Single Temperature Library Access
\item Calculate element AWR from isotope data
\item Multi-Temperature Library Access
\end{enumerate}
\end{xslide}

\addtime{360}
\begin{note}\small
\begin{enumerate}
\setcounter{enumi}{-1}
\item Create single temperature continuous energy \texttt{Library}; retrieve
  table meta-data by SZA.
\item Create \texttt{Chart} of the nuclides; calculate atomic weight
  ratio of elements using nuclide atomic weight ratios and abundances
  and compare.
\item Create multi-temperature continuous energy \texttt{Library};
  retrieve table meta-data by SZA and temperature.
\end{enumerate}

\end{note}

\begin{xslide}
\begin{center}\Large
Status
\end{center}
\begin{enumerate}
\item \sout{MG szaid $\Leftrightarrow$ CE szaid}
\item \textcolor{red}{Several user access through polyvalent program pattern}
\item \textcolor{green}{NDI multi-temperature features}
\item \textcolor{green}{NDI sza translation features}
\item \textcolor{green}{Curated Data}
\item \textcolor{green}{Continuous Energy Libraries}
\end{enumerate}

\end{xslide}

\addtime{60}
\begin{note}\small
\begin{enumerate}
\item MG elementals make one-to-one translation impossible; item six
  supersedes this requirement
\item OO C++ library API implemented; others may follow
\item Done
\item Done
\item Done
\item Done
\end{enumerate}
\end{note}

\begin{xslide}
\begin{center}\Large
To Do
\end{center}

\begin{itemize}
\item Path search policies for Exsdir
\item SZA and SZAID search policies in Library
\item Temperature match policy in Library
\item Flat C/FORTRAN interface (Requirement 2)
\item Command line interface (Requirement 2)
\item Physical Constants class
\end{itemize}
\end{xslide}

\addtime{60}
\begin{note}\small
\begin{itemize}
\item Code currently must ensure data in place and paths correct;
  NDI/xsdir search path policies needed.
\item Library currently looks for SZA in library list only and partial
  or complete SZAID in xsdir only; a half dozen other search policies
  are possible and could be implemented with appropriate templating.
\item Library currently looks for nearest temperature; a half dozen
  other search policies are possible and could be implemented with
  appropriate templating.
\item OO C++ API currently implemented; additional flat API could be
  defined and code written.
\item OO C++ API currently implemented; stand alone CLI application
  could be written.
\item constants.hh define a subset of physical constants in ndatk
  appropriate units; could define curated Constants class with user
  defined units.
\end{itemize}
\end{note}
\end{document}
